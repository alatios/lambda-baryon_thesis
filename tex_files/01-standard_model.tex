\chapter{Introduction to flavour physics}
Ever since Democritus' philosophy of atomism, one of the driving desires behind mankind's advancements in the fields of natural science has been to reduce reality to its basic components. [...]

[...], this chapter explores the theoretical framework for the rest of thesis, introducing key concepts such as \textit{spin}, \textit{helicity} and the importance of \textit{angular distributions} of decay products.

\section{The Standard Model of Particle Physics}
[Introduzione? Da qualche parte devi definire la sigla SM.]

\subsection{Elementary particles}
\begin{figure}[t!]
	\centering
	\includegraphics[scale=0.15]{graphics/01-standard_model/Standard_Model_of_Elementary_Particles.pdf}
	\caption[Currently known Standard Model elementary particles.]{The seventeen currently known elementary particles of the Standard Model. Antiparticles are not depicted.}
	\label{fig:particle_zoo}
\end{figure}

Intuitively, a particle is said to be \textit{elementary} when no substructure can be probed. 
A century of efforts in the fields of nuclear, quantum, and high energy physics has whittled down the spectrum of matter to just seventeen unique fundamental particles, colloquially known as the \textit{particle zoo} and depicted in Figure \ref{fig:particle_zoo}.

Each particle is joined by an \textit{antimatter particle} (\textit{antiparticle} for short), a companion of opposite charge identified by the prefix \textit{anti-}, e.g. antimuon for the muon; the only exception to this naming convention is the electron, whose antiparticle, for historical reasons, is known as positron.
While often omitted for the sake of brevity, antiparticles are elementary particles in every respect, distinct from their partners (bar the neutral gauge bosons, which are their own antiparticles) and related to them through the transformation of \textit{charge conjugation}.

\subsubsection{Leptons}
Leptons are fermions (half-integer spin particles) not sensitive to the strong nuclear interaction.
There are currently six \textit{flavours} of leptons grouped in three generations: each generation comprises a \textit{charged} lepton (electron, muon, tauon) and a \textit{neutral} lepton (electron neutrino, muon neutrino, tauon neutrino).

All charged leptons have a charge of $-q_e$, where $q_e$ is defined as the \textit{elementary positive charge}, and their mass ranges from $\approx \SI{0.5}{MeV}$ for the electron to over $\SI{1.7}{GeV}$ for the tauon.
By contrast, as the names suggest, all neutrinos are electrically neutral and are assumed massless in the Standard Model\footnote{The observation of flavour oscillation in solar neutrinos shows that neutrinos do in fact have non-zero, albeit very small, mass. This discrepancy is considered one of the major challenges to the Standard Model.}; this implies that their only meaningful interactions happen through the weak nuclear force, which grants them their characteristic evasiveness to most particle detectors.


\subsubsection{Quarks}
Much like leptons, quarks are also fermions existing in three generations. The main difference from the former category is that quarks, besides interacting through weak and electromagnetic forces, are also susceptible to the strong nuclear forces; this allows them to bind together in composite states known as \textit{hadrons}, which are classified as \textit{baryons} (states of three quarks) and \textit{mesons} (states of one quark and one antiquark).

Quarks can be classified as \textit{up-type} (up, charm and top quarks) and \textit{down-type} (down, strange and bottom quarks): up-type quarks have a fractionary charge of $+\frac{2}{3} q_e$, whereas down-type quarks have a charge of $-\frac{1}{3} q_e$. All quarks also have one of three \textit{color} charges (red, green or blue), while antiquarks similarly have one of three \textit{anti-color} charges (antired, antigreen or antiblue). A combination of all three colors/anti-colors or a combination of a color and its matching anticolor produces \textit{colorless} particles, a property of all observed quark composite states.

Unlike leptons, quarks are impossible to observe directly: according to the phenomenon of \textit{color confinement}, the energy of the interaction field between two color charges being pulled apart increases with their distance until it becomes high enough to create a quark-antiquark pair.
This process of \textit{fragmentation} develops many times over in such a way that the final observable state is entirely composed of colorless particles.
For this reason, high energy physics experiments such as LHCb do not detect free quarks, instead observing cone-shaped streams of hadrons known as \textit{hadronic jets}.

\subsubsection{Gauge bosons and fundamental interactions}
The fundamental forces driving the interactions between elementary particles are introduced in the Standard Model via the so-called \textit{gauge principle}.

[Gruppo di simmetria, principio di gauge, QCD e teoria elettrodebole.]

There is no gauge boson associated to the fourth known fundamental force, gravity.
Since every attempt to reconcile the general theory of relativity with quantum mechanics has failed so far, gravity is presently excluded from the Standard Model; this doesn't affect SM predictions at the subatomic level on account of the remarkably low intensity of said force, over 30 orders of magnitude lower than the weak interaction.

\subsubsection{The Higgs boson}
[Rottura spontanea della simmetria.]

\subsection{Flavour physics}
An unfamiliar reader may find it amusing to employ the term \textit{flavour} to refer to what have been so far presented as different kinds of particles altogether.
However quirky, the lexical choice highlights a defining feature: flavour, much like the degree of sweetness in a recipe, can change.

As often happens in particle physics, the rules are easier for leptons. For a given flavour, one can define a \textit{leptonic flavour number} $L_\ell$ as the sum total of the difference between the numbers of charged leptons and anti-charged leptons plus the difference between the numbers of neutrinos and antineutrinos of said flavour:
\begin{equation}
L_\ell
\coloneqq
n(\ell^-) - n(\ell^+)
+
n(\nu_\ell) - n(\bar{\nu}_\ell).
\end{equation}
For all three leptonic flavours, $L_\ell$ is conserved in every interaction except neutrino oscillations.

Quarks are not as straightforward.
A similarly defined quark flavour number, such as the so-called \textit{topness}
\begin{equation}
T
\coloneqq
n(t) - n(\bar{t}),
\end{equation}
is preserved through EM and strong interactions, but can change when the state undergoes a \textit{weak charged interaction}, i.e. a weak interaction mediated by the charged gauge bosons $W^\pm$.

[Mescolamento dei sapori.]

\begin{equation}
	\begin{pmatrix}
		d' \\
		s' \\
		b'
	\end{pmatrix}
	=
	\begin{pmatrix}
		V_{ud} & V_{us} & V_{ub} \\
		V_{cd} & V_{cs} & V_{cb} \\
		V_{td} & V_{ts} & V_{tb}
	\end{pmatrix}
	\begin{pmatrix}
		d \\
		s \\
		b
	\end{pmatrix}
	\label{eq:CKM_matrix}
\end{equation}

[Probabilità oscillazione propto Vqq'. Riparametrizzazione con $\delta$.]

The phase $\delta$ is known as the CP-violating phase. To fully understand what it means and its role in particle physics, however, we first have to talk about discrete symmetries. 

\section{Discrete symmetries and CP violation}
[Simmetrie in meccanica quantistica. Tre tipologie di simmetrie: continue, di campo e discrete. Sì, insomma, questa parte sono le lezioni di Giammarchi da riscaricarsi. Ops.]

\subsection{Parity}
[Descrizione, parità intrinseca, violazione e primo esperimento.]

\subsection{Charge conjugation}
[Descrizione, C-parità intrinseca, violazione e primo esperimento.]

\subsection{Time inversion}
[Descrizione, violazione e primo esperimento.]

\subsection{CP symmetry}
[Esempio di soddisfazione di CP. Violazione della simmetria CP. Teorema CPT.]

\section{Electromagnetic dipole moments}

\subsection{Spin}
[Puoi axare questa parte, se vuoi.]

[Anche se ho usato la parola con leggerezza nei capitoli precedenti], the concept of \textit{spin} may very well be one of the most challenging in particle physics.

\subsection{EDM and MDM}
[Comportamento sotto CPT, evidenza che momento non nullo implica violazione CP.]

\section{Helicity formalism}
[Il Richman.]

\section{The \texorpdfstring{$\Lambda$}{Lambda} baryon}