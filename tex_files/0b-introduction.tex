\chapter*{Introduction}
\addcontentsline{toc}{chapter}{Introduction}
\markboth{Introduction}{}

%Cose da dire:
%
%* SM ha buchi
%* EMDM sondano quei buchi
%* Come vogliamo misurare EMDM noi
%* Come è ripartita la tesi

From the discovery of dark matter in spiral galaxies to the confirmation of neutrino oscillation as solution to the solar neutrino problem, evidence has piled up in favour of the incompleteness of the Standard Model of Particle Physics, currently the best description of particles at the subatomic level.
The subject of violation of the CP discrete symmetry has gained traction in recent years due to the ${10}^{10}$ factor disparity between the known Standard Model sources of violation and the extent required to explain the present matter-antimatter asymmetry in the Universe. 

One promising pathway to new physics is the study of electromagnetic dipole moments of elementary and composite particles.
Permanent electric dipole moments (EDMs) introduce a CP-violating term in the system's Hamiltonian;
given that expected Standard Model contributions are orders of magnitude smaller than current experiment sensitivities, EDM upper limits place strict constraints on the existence of new physics.
Magnetic dipole moments (MDMs) can further be used to probe violation of the CPT theorem, which predicts MDMs to be identical in magnitude and opposite in sign for particles and matching antiparticles.

Electromagnetic dipole moments of long-lived particles can be measured from the precession of their spin-polarization vector in a strong magnetic field, which depends on the particle's gyroelectric and gyromagnetic factors.
In this thesis, I present my work in preparation of a measurement of the electromagnetic dipole moments of the \lz baryon with the LHCb experiment.
Long-lived \lz baryons from the exclusive \demonstratorfull decay are selected with the requirement that the \lz decay after the LHCb dipole magnet, allowing for the comparison of initial and final polarization states.
Theoretical background for the EDM/MDM measurement approach and specifics on the LHCb detecting apparatus are reported in Chapters \ref{cap:flavour_physics} and \ref{cap:LHCb} respectively.

%Electric and magnetic dipole moments of particles are sensitive to physics within and beyond the Standard Model. In this thesis, sensitivity studies for the measurement of the Lambda baryon electromagnetic dipole moments based on pseudo experiments will be performed. In addition, the possibility of a first measurement using data collected with the LHCb detector will be explored. 

For the first part of my thesis, detailed in Chapter \ref{cap:vertex_reconstruction}, I report on my work in understanding and improving the vertex reconstruction process in LHCb, with the main goal of mitigating the low efficiency of \lambdadecay reconstruction in \demonstratorshort decays.
I also analyze the $z$ coordinate resolution of the reconstructed \lz vertex to gauge possible sources of bias.

In the second part of my thesis, I focus on the development and finalization of the three major steps in the signal selection process: preliminary filters, rejection of \physbkgshort physical background,
%(including a newly-introduced \kshort veto based on the Armenteros-Podolanski technique)
and discrimination of signal through the training and testing of a supervised learning multivariate classifier.
Results on this front are collected in Chapter \ref{cap:event_selection}.

Finally, in Chapter \ref{cap:angular_distribution} I capitalize on my earlier work to perform a first analysis of the angular distribution of \lambdadecay decay products, a key stepping stone in the prospective measurement of the \lz electromagnetic dipole moments.