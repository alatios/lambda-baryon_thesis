\chapter*{Conclusions}
\addcontentsline{toc}{chapter}{Conclusions}
\markboth{Conclusions}{}

Electric and magnetic dipole moments of particles are sensitive to physics within and beyond the Standard Model.
In this thesis, I worked on various aspects of the \demonstratorshort decay analysis in preparation of a first measurement of the \lz electromagnetic dipole moments using the LHCb Run 2 dataset.

Less than half of candidate \lambdadecay events reach convergence in the vertex reconstruction process.
I conducted topological studies on a sample of simulated events to show that this is a result of a conflict of information in $xz$ (bending) and $yz$ (non-bending) track propagation planes.
Through further investigation in the measured kinematic variables and comparison with the Monte-Carlo-generated values, I exposed a systematic underestimation of $p_z$ in pion tracks reconstructed from hits in the T1--T3 downstream tracking stations.
Said bias is only observed in non-converging \lambdadecay events and is understood to play a role in the $xz$-$yz$ discrepancy at the origin of the vertexing failure.
Additional research is currently underway to locate and fix the source of $p_z$ bias, starting with the track momentum fit process at T station level.

For the time being, I demonstrated that recovery of a significant percentage of failed events in possible by modifying the main vertex fitting algorithm to increase the weight of track propagation in a specific plane.
A threefold refit approach, attributing more importance to $yz$, $xz$ and $xy$ planes sequentally, results in a $+26.4\%$ increase in signal statistics.
Comparisons to Monte Carlo truth reveal that recovered events have suboptimal reconstruction, with a median bias on the $z$ component of the \lambdadecay vertex \SI{20}{\centi\meter} greater than standard reconstructed events.
Studies confirm that this is due to poor track information available in these events;
the impact of lower vertex resolution on the \lz electromagnetic dipole moment measurement will have to be evaluated in future analyses.

Working on \demonstratorshort signal selection, I finalized the three main steps of the process:
loose preliminary selections, including requirement of Decay Tree Fitter convergence with \jpsi and \lz mass constraints;
rejection of \physbkgshort physical background with an invariant mass veto and a cut in the Armenteros-Podolanski $\alpha$--\pt space;
the final selection of signal with a histogram-based gradient boosting classification tree, trained with simulated signal and LHCb combinatorial background and optimized to maximize \demonstratorshort signal significance.
The $m(J/\psi\,\Lambda^0)$ invariant mass fit after all steps shows excellent agreement with data, estimating a signal (background) yield of $3590 \pm 60$ ($2420 \pm 50$).

%% lz bias after selection
