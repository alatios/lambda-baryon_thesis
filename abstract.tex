\documentclass[12pt,a4paper]{book}
\usepackage[T1]{fontenc}
\usepackage{textcomp}		% Companion symbols (useful to have)
\usepackage[utf8]{inputenc}
\usepackage[english]{babel}
\usepackage{lmodern}		% Uses the superior font

\usepackage{pdf14}
%\usepackage{pdfx}			% Let's PDF/A it from the start
\usepackage{graphicx}		% For (you guessed it) including graphics

%% Cover-page-related
\usepackage[vcentering,dvips,left=4.0cm,right=3.0cm,top=3.5cm,bottom=3.0cm,includeheadfoot]{geometry}
%\geometry{textwidth=390pt}

%% Bibliography-related
\bibliographystyle{unsrt}
\usepackage{tocbibind}		% Adds bibliography (as well as figure and table lists)
							% to table of contents.

%% For better captions
\usepackage[font=small,labelfont=bf]{caption}
\usepackage{subcaption}

%% For SI units
\usepackage{siunitx}

%% Math related
\usepackage{amsmath}	% Matrix environments
\usepackage{amssymb} % \lesssim
\usepackage{mathtools} % Coloneqq, xrightarrow
\usepackage{braket} % Take a guess
\usepackage{dsfont} % For identity operator
\usepackage{bm}		% For bold math

%% For better tables (see https://people.inf.ethz.ch/markusp/teaching/guides/guide-tables.pdf)
\usepackage{booktabs}
\renewcommand{\arraystretch}{1.2}

%% This avoids page breaks in footnotes
\interfootnotelinepenalty=10000
%% This makes footnote numbering continue instead of resetting at each chapter
\counterwithout{footnote}{chapter}

%% To change the header
\usepackage{fancyhdr}
\pagestyle{fancy}

%% These change the display of \leftmark and \rightmark, for some reason
%% No, I have no idea what they do atm. Truly shameful
\renewcommand{\chaptermark}[1]{\markboth{#1}{}}
\renewcommand{\sectionmark}[1]{\markright{\thesection~~#1}}

\fancyhf{}

\setlength{\headheight}{15pt} %% Changes header height to accomodate text

%% E/O = even/odd pages
%% L/C/R = left/center/right alignment
%% H/F = header/footer
%% \thepage is page number
%% \leftmark is the chapter title
%% \rightmark is the section number and title
%% \nouppercase because otherwise bibliography comes up as BIBLIOGRAPHY in the header
\fancyhead[ELH]{\thepage}
\fancyhead[ORH]{\thepage}
%\fancyhead[ERH]{\leftmark}
\fancyhead[ERH]{\nouppercase{\leftmark}}
%\fancyhead[OLH]{\rightmark}
\fancyhead[OLH]{\nouppercase{\rightmark}}

%% This automatically makes math mode bold in chapter titles, I believe
\makeatletter
\g@addto@macro\bfseries{\boldmath}
\makeatother

%% Barn has been deprecated :( Fuck you Fermi
\DeclareSIUnit\barn{b}
\DeclareSIUnit\mev{\mega\electronvolt}
\DeclareSIUnit\gev{\giga\electronvolt}
\DeclareSIUnit\tev{\tera\electronvolt}
\DeclareSIUnit\rad{rad}
\DeclareSIUnit\mrad{\milli\rad}

%% For trailing space after shortcuts (otherwise it sticks to the next word)
\usepackage{xspace}
%% More shortcuts
\newcommand{\demonstratorfull}{$\Lambda_b^0 \rightarrow J/\psi~(\rightarrow \mu^+ \mu^-)~\Lambda^0~(\rightarrow p\pi^-)$\xspace}
\newcommand{\demonstratorshort}{$\Lambda_b^0 \rightarrow J/\psi\,\Lambda^0$\xspace}
\newcommand{\physbkgfull}{$B^0 \rightarrow J/\psi~(\rightarrow \mu^+ \mu^-)~K^0_S~(\rightarrow \pi^+\pi^-)$\xspace}
\newcommand{\physbkgshort}{$B^0 \rightarrow J/\psi\,K^0_S$\xspace}
\newcommand{\lz}{$\Lambda^0$\xspace}
\newcommand{\lbz}{$\Lambda_b^0$\xspace}
\newcommand{\jpsi}{$J/\psi$\xspace}
\newcommand{\pim}{$\pi^-$\xspace}
\newcommand{\bz}{$B^0$\xspace}
\newcommand{\kshort}{$K^0_S$\xspace}
\newcommand{\slab}{$\text{S}_\text{L}$\xspace}
\newcommand{\shad}{$\text{S}_H$\xspace}
\newcommand{\slambda}{$\text{S}_\Lambda$\xspace}
\newcommand{\slambdal}{$\text{S}_{\Lambda\text{L}}$\xspace}
\newcommand{\lambdadecay}{$\Lambda^0 \rightarrow p\pi^-$\xspace}
\newcommand{\kshortdecay}{$K^0_S \rightarrow \pi^+\pi^-$\xspace}
\newcommand{\pt}{$p_\text{T}$\xspace}
\newcommand{\pl}{$p_\text{L}$\xspace}
\newcommand{\ptstar}{$p_\text{T}^\star$\xspace}
\newcommand{\plstar}{$p_\text{L}^\star$\xspace}
\newcommand{\thetap}{$\theta_p$\xspace}
\newcommand{\cthetap}{$\cos\theta_p$\xspace}
\newcommand{\phip}{$\phi_p$\xspace}

%% Math operators
\DeclareMathOperator{\sign}{sign}
\DeclareMathOperator{\arctantwo}{arctan2}

%%%%%%%%%%%%%%%%%%%%%%%%%%%%%%%%%%%%%%%%%%%%%%%%%%%%%%%%%%%%%%%%

\begin{document}

%%%%%%%%%%%%%%%%%%%%%%%%%%%%%%%%

%\frontmatter

%% Cover page
%\newgeometry{centering}	% Make the page centered on paper
%%%\title{\textsc{Verso la misura del flusso di neutrini da CNO: Analisi radiale del fondo $^{210}$Bi-$^{210}$Po nel rivelatore Borexino}}
%\author{Alessandro De Gennaro}
%\date{2018}


\begin{titlepage}
	\begin{figure}[t]
		\centering
		\includegraphics[width=390pt]{graphics/cover-page/logo.jpg}
		\centering
	%	\vspace{0.1 cm}
	\end{figure}	
\begin{center}
{\large Corso di Laurea Magistrale in Fisica}
\end{center}

\begin{center}
\vspace{2 cm}
{\Large \textsc{A study for the measurement of the $\Lambda$ baryon electromagnetic
dipole moments in LHC}b\par}
\end{center}
%\par
  \vspace{2 cm}
  
  \begin{flushleft}
  		 Relatore: \hskip 0.62 cm Prof. Nicola NERI\\
		 
  		 \noindent Correlatore: \hskip 0.1 cm Dott.ssa\ Elisabetta SPADARO NORELLA
  \end{flushleft}
  \vspace{1 cm}
  \begin{flushright}
  	Tesi di Laurea di:\\ Alessandro DE GENNARO\\ Matricola \hskip 0.1 cm 933289\\ Codice P.A.C.S.: \hskip 0.1 cm 14.20.-c
  \end{flushright}
    	  
%\vfill
\begin{center}
\vspace{2 cm}
{\large Anno Accademico 2020-2021}
\end{center}
\end{titlepage}
%\restoregeometry		% Restore the page layout as earlier

%% Bring order to chaos. \chaptermark fixes uppercase letters in header
%\tableofcontents
%\chaptermark{Contents}
%\listoffigures
%\chaptermark{List of Figures}
%\listoftables 
%\chaptermark{List of Tables}

%%%%%%%%%%%%%%%%%%%%%%%%%%%%%%%%

\mainmatter

\chapter*{Abstract}

Electric and magnetic dipole moments (EDMs and MDMs) of elementary and composite particles are a powerful tool to probe physics beyond the Standard Model:
permament EDMs are a potential signature of new sources of CP symmetry violation, while MDMs provide precision measurements of QCD predictions and can be used to test the validity of the CPT theorem.

One approach to study EDMs and MDMs is to exploit the precession of the particle's spin %they trigger when flying
through a magnetic field;
this is achieved by comparing initial and final polarization states of a sample of particles through a fit to the angular distribution of their decay products.
This technique is ripe for application at LHCb, a single-arm spectrometer designed to study heavy-flavour physics using proton-proton collisions at the Large Hadron Collider.
In this thesis, I analyzed long-lived \lz baryon decays from the exclusive \demonstratorfull channel in preparation of the first measurement of \lz electromagnetic dipole moments with the \SI{6}{\per\femto\barn} LHCb Run 2 dataset.

The first part of my thesis was dedicated to the vertex reconstruction of \lbz and \lz decays.
Vertexing efficiency falls consistently below 50\% for long-lived \lz decaying after the LHCb dipole magnet, halving the potential signal yield.
%Through topological studies of the events, I found this to be a result of a conflict of track information in $xz$ (bending plane) and $yz$ (no bending) planes, with the leading explanation being a systematic underestimation of $p_z$ in pion tracks.
Through topological studies of the events, I found this to be a result of the discrepancy in $z$ coordinate between the point of closest $p\pi^-$ track distance in the $xz$ (bending) plane and the crossing track point in the $yz$ (non-bending) plane.
To mitigate the problem, I built and deployed an alternative vertex fitting algorithm that increases the weight of specific track propagation planes.
Refitting \lambdadecay decays with the new algorithm results in a $+26.4\%$ increase in signal statistics, recovering a quarter of the previously non-converging events.
The reason for the $xz$-$yz$ inconsistency is currently unknown;
however, an observed systematic underestimation of $p_z$ for pion tracks in non-converged events suggests it could be linked to poor momentum measurement at the T tracking stations.

In the second part of my thesis, I finalized a three-step signal selection process based on loose preliminary filters, rejection of \physbkgfull physical background, and discrimination of \demonstratorshort events with a histogram-based gradient boosting classification tree.
The $m(J/\psi\,\Lambda^0)$ invariant mass fit after all steps shows excellent agreement with data, estimating a signal (background) yield of $3590 \pm 60$ ($2420 \pm 50$) in the $\pm3\sigma$ region around the \lbz resonance peak.

Finally, I performed a first analysis of the angular distribution of \lz decay products from \demonstratorshort events.
Angular reconstruction of \thetap (polar) and \phip (azimuthal) proton production angles in the \lz helicity frame is unbiased net of acceptance effects.
Angular resolutions, defined as the root mean square deviation of proton angles from their generated values, span 0.2--0.3 for \cthetap and 1.0--1.2 for \phip;
in both cases the results are acceptable, being within roughly one sixth of the allowed variable ranges.
Reconstructed signal events have a median \SI{14}{\centi\meter} positive bias on the $z$ component of the \lambdadecay vertex, which negatively affects proton angular resolutions.
%Resolutions heavily worsen as function of the bias on the $z$ component of the \lambdadecay vertex.
This bias is mostly attributable to <<ghost vertex>> events, where $p\pi^-$ tracks are bent by the magnet into a second crossing point misidentified as the apparent production vertex by the algorithm.
%Removal of this class of events, motivated by encouraging early results with changes to the vertex reconstruction algorithm, improves proton angular resolutions by a factor 2--3 across the full range of values.
Ongoing tests with a modified vertex reconstruction algorithm show encouraging results in reducing the number of ghost vertex events;
the complete solution of this problem would improve proton angular resolutions by a factor 2--3 across the full range of values.

Identified issues in this analysis do not compromise the prospective measurement of \lz electromagnetic dipole moments.
On the contrary, the achieved signal yield and absence of bias in proton angular distributions confirm that competitive results with long-lived \lz baryons are possible with Run 2 data.
Given the upcoming statistics increase expected for Run 3 and the boost in yield and resolution an improved vertexing algorithm would provide, the outlook is promising for a first measurement of \lz EDMs and MDMs at LHCb.

%%%%%%%%%%%%%%%%%%%%%%%%%%%%%%%%%%%%%%%%%%%%%%%%%%%%%%%%%%%%%%%
%\noindent\rule{\textwidth}{0.4pt}
%%%%%%%%%%%%%%%%%%%%%%%%%%%%%%%%%%%%%%%%%%%%%%%%%%%%%%%%%%%%%%%

%Simulated \demonstratorshort events passing the full selection process retain a median \SI{14}{\centi\meter} bias in the $z$ component of the reconstructed \lambdadecay vertex, which has a detrimental effect on \cthetap and \phip resolutions.
%This can mostly be attributed to proton and pion tracks being bent by the magnetic field into a second downstream crossing point, acting as local $\chi^2$ minimum during the vertexing process and being erroneously selected as the \lz decay vertex.
%Removing this class of events (31.6\% of the simulated sample) improves proton angular resolutions by a factor 2--3 across the full range of values.
%Changing the vertex fitting algorithm to account for multiple $\chi^2$ minima would therefore significantly affect the dipole moment measurement and must be considered a high priority for the analysis.

\end{document}
