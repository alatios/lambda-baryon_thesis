\chapter{The LHCb experiment}
\label{cap:LHCb}

\section{The Large Hadron Collider}
The Large Hadron Collider (LHC for short) is the largest and most powerful particle collider in the world.

\section{The LHCb experiment and detector}
LHCb (the \textit{b} stands for \textit{beauty}\footnote{Before settling on the names \textit{top} and \textit{bottom} for the third generation of quarks, the names \textit{truth} and \textit{beauty} were among those proposed. While they never gained enough momentum in the scientific community, echoes of the failed nomenclature are still present in heavy quark vocabulary, for instance in the alternative name \textit{truth} for the \textit{topness} flavour number mentioned in Section \ref{sec:flavour-physics}, as well as in the official name for the LHCb experiment.}) is one of the four main experiments at the LHC.

Elenca i successi.

\label{info:LHCb_system}
RICORDA DI DIRE IL SISTEMA DI COORDINATE!
Non usare questo, che è preso da online:
A right-handed coordinate system is defined centred on the interaction point, with z along the beam axis and y pointing upwards.

\subsection{Tracking}

\subsubsection{VELO}

\subsubsection{Tracker Turicensis}

\subsubsection{T stations}

\subsubsection{Track classification}
Qui tutta la storia delle T track

\subsection{Particle identification}

\subsubsection{RICH}

\subsubsection{Calorimeter}

\subsubsection{Muon system}

\section{The LHCb data flow}
Qui trigger, track reconstruction e tutto il resto.

\section{LHCb detector upgrade for Run 3}

\section{Data}
Non so se vada qui ma da qualche parte deve andare.