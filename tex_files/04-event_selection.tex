%% Problema: deve essere grassetto nella ToC e nel titolo del capitolo, ma non grassetto nell'header.
\chapter{Signal event selection}
\label{cap:event_selection}

\section{Prefiltering}
\label{sec:prefilter}
\textit{Prefilters}, also refered to as \textit{preliminary selections} or simply \textit{pre-selections}, are the foundation of the signal selection process.
The main objective of this step is to improve the signal-to-background ratio and reduce the computational workload to analyze data with cuts on kinematic variables.

\begin{table}[t]
	\begin{center}
	\begin{tabular}{|l|c||c|c|}
		\hline
		Variable & Unit & Minimum & Maximum \\
		\hline
		\hline
		$p(p)$ 						& MeV/$c$ 	& 2\,000	& 500\,000 \\
		$p_T(p)$ 					& MeV/$c$ 	& 400		& -- \\
		$p(\pi^-)$ 					& MeV/$c$ 	& 10\,000	& 500\,000 \\
		$z_\Lambda^\text{vtx}$		& mm		& 5\,500	& 8\,500 \\
		$p_T(\Lambda^0)$ 			& MeV/$c$ 	& 450		& -- \\
		$m(p\pi^-)$	(Vertex Fitter)	& MeV/$c^2$	& 600		& 1\,500 \\
		$m(p\pi^-)$	(combined)		& MeV/$c^2$	& --		& 2\,000 \\
		$m(p\pi^-)$	(measured)		& MeV/$c^2$	& --		& 1\,500 \\
		$\cos\xi_p (\Lambda^0)$		& --		& 0.9999	& -- \\
		$\Delta \chi^2_\text{PV} (\Lambda^0)$
									& -- 		& --		& 200 \\
		$\chi^2_\text{dist} (\Lambda^0)$
									& --		& --		& $2\times{10}^{7}$ \\
		$\chi^2_\text{vtx} (\Lambda^0)$
									& --		& --		& 750 \\
		$\lvert m(\mu^+ \mu^-) - m_\text{PDG} (J/\psi) \rvert$
									& MeV/$c^2$ & --		& 90 \\
		$m(J/\psi~\Lambda^0)$ (combined)
									& MeV/$c^2$	& 4\,700		& -- \\
		$m(J/\psi~\Lambda^0)$ (Vertex Fitter)
									& MeV/$c^2$	& --		& 8\,500 \\
		$\lvert \cos\xi_p (\Lambda^0_b) \rvert$
									& --		& 0.99		& -- \\
		$\Delta \chi^2_\text{PV} (\Lambda^0_b)$
									& -- 		& --		& 1\,750 \\
		$\chi^2_\text{vtx} (\Lambda^0_b)$
									& --		& --		& 150 \\
		\hline
	\end{tabular}
	\end{center}
	\caption{Prefilter selection criteria applied to simulated \demonstratorshort signal and Run 2 data. The Vertex Fitter invariant mass is computed by the homonymous algorithm; the \textit{combined} invariant mass is computed from the 4-momenta of the daughter particles at the first measurement position, without track extrapolation; the \textit{measured} invariant mass is computed in the same way as the combined mass, but after extrapolation at the reconstructed decay vertex of the mother particle. Angle $\xi_p$ for a particle is computed between the line connecting its origin and decay vertices and the direction of its momentum. $\Delta \chi^2_\text{PV} (\Lambda^0)$ is the increase of the primary vertex $\chi^2$ when the particle is included in the fit. $\chi^2_\text{dist}$ is the geometrical distance between the primary vertex and the particle decay vertex in $\chi^2$ units.}
	\label{tab:4:prefilters}
\end{table}

\begin{figure}[t]
	\centering
	\begin{subfigure}{.45\textwidth}
		\includegraphics[width=\textwidth]{graphics/04-event_selection/LEVz_left.pdf}
		\caption{}
	\end{subfigure}
	\begin{subfigure}{.45\textwidth}
		\includegraphics[width=\textwidth]{graphics/04-event_selection/LEVz_right.pdf}
		\caption{}
	\end{subfigure}
	\caption{Efficiency of the $z_\text{vtx}^\Lambda \geq z_\text{cut}^\text{left}$ \textit{(a)} and $z_\text{vtx}^\Lambda \leq z_\text{cut}^\text{right}$ \textit{(b)} prefilter selection criteria on \demonstratorshort simulated signal, as function of the respective thresholds. The \textit{dotted vertical lines} mark the chosen thresholds.}
	\label{fig:4:z_lambda_cuts}
\end{figure}

The applied prefilter criteria are listed in Table \ref{tab:4:prefilters}.
The most impactful selection is the one applied to $z_\text{vtx}^\Lambda$, the $z$ component of the \lambdadecay decay vertex;
the efficiencies for the left and right cuts as a function of the threshold, estimated on simulated signal, are shown in Figure \ref{fig:4:z_lambda_cuts}.
Since we require $\Lambda^0$ to decay after the dipole magnet in order to observe spin precession, the [@todo] of the $z_\text{vtx}^\Lambda \geq \SI{5.5}{\meter}$ cut cannot be avoided.
Other selections have a much lower impact on signal, with efficiencies $\gtrsim 80\%$, resulting in a total prefilter efficiency of [@todo].

In addition to the kinematic selections, [DTF FixJPsiLambda parte del prefilter]

\begin{figure}[t]
	\centering
	\begin{subfigure}{.45\textwidth}
		\includegraphics[width=\textwidth]{graphics/04-event_selection/Lambda_endvertex_z_true.pdf}
		\caption{}
	\end{subfigure}
	\begin{subfigure}{.45\textwidth}
		\includegraphics[width=\textwidth]{graphics/04-event_selection/Lambda_endvertex_z.pdf}
		\caption{}
	\end{subfigure}
	\caption{Distribution of true \textit{(a)} and reconstructed \textit{(b)} $z_\text{vtx}^\Lambda$ in simulated \demonstratorshort signal events, without (\textit{dark grey}) and with (\textit{light grey}) prefiltering.}
\end{figure}

\begin{figure}[t]
	\centering
	\begin{subfigure}{.45\textwidth}
		\includegraphics[width=\textwidth]{graphics/04-event_selection/Lambda_endvertex_z_vs_y_true.pdf}
		\caption{}
	\end{subfigure}
	\begin{subfigure}{.45\textwidth}
		\includegraphics[width=\textwidth]{graphics/04-event_selection/Lambda_endvertex_z_vs_y.pdf}
		\caption{}
	\end{subfigure}
	\caption{Distribution of simulated \demonstratorshort signal events (prefilters applied) with $z_\Lambda^\text{VF} \geq \SI{7.0}{\meter}$, as function of true (\textit{left}) and reconstructed (\textit{right}) $y_\Lambda^\text{vtx}$ and $z_\text{vtx}^\Lambda$. This corresponds to a side view of true and reconstructed \lz decay vertices.}
\end{figure}

\subsection{Reconstruction of \texorpdfstring{\lz}{Lambda} decay vertex}
\label{sec:lambda_endvertex_bias}
\begin{figure}[t]
	\centering
	\begin{subfigure}{.45\textwidth}
		\includegraphics[width=\textwidth]{graphics/04-event_selection/Lambda_endvertex_bias_z.pdf}
		\caption{}
	\end{subfigure}
	\begin{subfigure}{.45\textwidth}
		\includegraphics[width=\textwidth]{graphics/04-event_selection/Lambda_endvertex_bias_z_log.pdf}
		\caption{}
	\end{subfigure}
	\caption{Distribution of $z_\text{vtx}^\Lambda$ bias for simulated \demonstratorshort events in linear \textit{(a)} and logarithmic \textit{(b)} scales, without (\textit{dark grey}) and with (\textit{light grey}) prefiltering.}
\end{figure}

Confronto con Figure \ref{fig:2:t_station_top}.

\begin{figure}[t]
	\centering
	\begin{subfigure}{.45\textwidth}
		\includegraphics[width=\textwidth]{graphics/04-event_selection/bump_Lambda_true_endvertex_z_vs_x.pdf}
		\caption{}
	\end{subfigure}
	\begin{subfigure}{.45\textwidth}
		\includegraphics[width=\textwidth]{graphics/04-event_selection/bump_scatter_Lambda_endvertex_z_vs_x.pdf}
		\caption{}
	\end{subfigure}
	\caption{Distribution of simulated \demonstratorshort events (prefilters applied) with $z_\Lambda^\text{VF} - z_\Lambda^\text{true} \geq \SI{2.0}{\meter}$ as function of true \textit{(a)} and reconstructed \textit{(b)} $x_\Lambda^\text{vtx}$ and $z_\text{vtx}^\Lambda$. This corresponds to a top view of true and reconstructed \lz decay vertices.}
\end{figure}


\begin{figure}[t]
	\centering
	\begin{subfigure}{.45\textwidth}
		\includegraphics[width=\textwidth]{graphics/04-event_selection/Lambda_horizontality_bias.pdf}
		\caption{}
	\end{subfigure}
	\begin{subfigure}{.45\textwidth}
		\includegraphics[width=\textwidth]{graphics/04-event_selection/lambda_endvertex_z_bias_vs_horizontality_bias.pdf}
		\caption{}
	\end{subfigure}
	\caption{\textit{(a)} Horizontality bias distribution for simulated \demonstratorshort events (prefilters applied), comparing results from Decay Tree Fitter algorithm with \jpsi and \lz mass constraints with true values. \textit{(b)} Distribution of $z_\text{vtx}^\Lambda$ bias for events with horizontality bias $<1$ (\textit{horizontal hatching}) and $\geq 1$ (\textit{diagonal hatching}).}
\end{figure}

Qui devi menzionare l'orizzontalità, perché vi faccio riferimento nel Cap. 3. Devi dire che c'è il problema con grafico.

\section{Physical background veto}
\label{sec:B0_veto}
\begin{figure}[t]
	\centering
	\begin{subfigure}{.45\textwidth}
		\includegraphics[width=\textwidth]{graphics/04-event_selection/phys_bkg_lambda_comparison.pdf}
		\caption{}
	\end{subfigure}
	\begin{subfigure}{.45\textwidth}
		\includegraphics[width=\textwidth]{graphics/04-event_selection/phys_bkg_lambdab_comparison.pdf}
		\caption{}
	\end{subfigure}
	\caption{Comparison of simulated $m(p\pi^-)$ \textit{(a)} and $m(J/\psi~\Lambda^0)$ \textit{(b)} distributions: \demonstratorshort signal is labeled by \textit{horizontal hatching}, \physbkgshort physical background with $\pi^+ \rightarrow p$ mass hypothesis by \textit{diagonal hatching}.}
\end{figure}

\begin{figure}[t]
	\centering
	\begin{subfigure}{.45\textwidth}
		\includegraphics[height=.2\textheight]{graphics/04-event_selection/phys_veto_efficiencies.pdf}
		\caption{}
	\end{subfigure}
	\begin{subfigure}{.45\textwidth}
		\includegraphics[height=.2\textheight]{graphics/04-event_selection/phys_veto_sig_efficiencies_per_bin.pdf}
		\caption{}
	\end{subfigure}
	\caption[Efficiency of the physical background veto as a function of the invariant mass discrepancy threshold and of $J/\psi~\Lambda^0$ invariant mass bins.]{\textit{(a)} Efficiency of physical background veto as a function of the invariant mass discrepancy threshold on simulated signal (\demonstratorshort, solid) and background (\physbkgshort with proton mass hypothesis, dashed) events. Chosen threshold marked by dotted line. \textit{(b)} Efficiency of the veto on different $m(J/\psi~\Lambda^0)$ bins for \demonstratorshort signal events).}
\end{figure}

\section{HBDT classifier}
\label{sec:HBDT}

\subsection{Training data}

\begin{figure}[t]
	\centering
	\includegraphics[width=.6\textwidth]{graphics/04-event_selection/sig_bkg_distribution_balance.pdf}
	\caption{Signal (\textit{horizontal hatching}) and background (\textit{diagonal hatching}) data samples used for training the HBDT classifier. Test samples are taken from the same pool in 1:9 ratio.}
	\label{fig:4:HBDT_training_data}
\end{figure}

\subsection{Hyperparameter optimization and performance test}
\begin{figure}[t]
	\centering
	\begin{subfigure}{.45\textwidth}
		\includegraphics[width=\textwidth]{graphics/04-event_selection/confmatrix_train.pdf}
		\caption{}
	\end{subfigure}
	\begin{subfigure}{.45\textwidth}
		\includegraphics[width=\textwidth]{graphics/04-event_selection/confmatrix_test.pdf}
		\caption{}
	\end{subfigure}
	\caption{Confusion matrices visualizing the performance of the HBDT classifier on training \textit{(a)} and testing \textit{(b)} data samples. Percentages and chromatic scale are normalized to the true event classification: for instance, the top left and top right quadrants of a matrix represent the fraction of true background events reconstructed as background or signal, respectively. Binary classification uses an illustrative response threshold $s_\text{thres} = 0.5$.}
\end{figure}

\begin{figure}[t]
	\centering
	\begin{subfigure}{.45\textwidth}
		\includegraphics[width=\textwidth]{graphics/04-event_selection/sig_train_vs_test.pdf}
		\caption{}
	\end{subfigure}
	\begin{subfigure}{.45\textwidth}
		\includegraphics[width=\textwidth]{graphics/04-event_selection/bkg_train_vs_test.pdf}
		\caption{}
	\end{subfigure}
	\caption{Response distribution of the HBDT classifier on signal \textit{(a)} and background \textit{(b)} events. The training sample is represented by horizontal hatching, the test sample by diagonal hatching.}
\end{figure}

\begin{figure}
	\centering
	\includegraphics[width=.6\textwidth]{graphics/04-event_selection/roc.pdf}
	\caption{Receiving operating characteristic (ROC) curve for the HBDT classifier on training (\textit{solid}) and test (\textit{dashed}) samples. The legend includes the area-under-curve (AUC) score.}
\end{figure}

\subsection{Threshold optimization}

\begin{figure}
	\centering
	\includegraphics[width=.6\textwidth]{graphics/04-event_selection/HBDT_signal_significance.pdf}
	\caption{Projected \demonstratorshort signal significance over background as a function of the HBDT response threshold used for selection.}
\end{figure}

\section{Performance on data}
Gli invariant mass fits, essenzialmente.

\begin{figure}[t]
	\centering
	\begin{subfigure}{.45\textwidth}
		\includegraphics[width=\textwidth]{graphics/04-event_selection/MC_lambdab_hard_fit.pdf}
		\caption{}
	\end{subfigure}
	\begin{subfigure}{.45\textwidth}
		\includegraphics[width=\textwidth]{graphics/04-event_selection/data_lambdab_hard_fit.pdf}
		\caption{}
	\end{subfigure}
	\caption{Fitted $m(J/\psi~\Lambda^0)$ invariant mass distributions for simulated \demonstratorshort events \textit{(a)} and Run 2 data \textit{(b)} after all selection steps. Signal fit function is \textit{dashed}, background fit function in \textit{(b)} is \textit{dash-dotted}. The current best measurement for $\Lambda_b^0$ mass is marked by the \textit{dotted vertical line}. Fit pulls (data-fit discrepancy divided by uncertainty) are shown below the main plots.}
\end{figure}
