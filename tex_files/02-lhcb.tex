\chapter{The LHCb experiment}
\label{cap:LHCb}

\section{The Large Hadron Collider}

\begin{figure}[t]
	\centering
	\includegraphics[width=.6\textwidth]{graphics/02-lhcb/lhc_diagram.png}
	\caption[LHC schematic layout.]{Layout of the Large Hadron Collider with its four main experiments \cite{doi:10.1146/annurev-nucl-102010-130438}.}
	\label{fig:2:lhc_diagram}
\end{figure}

At the moment of writing, the Large Hadron Collider (LHC for short) is the largest and most powerful particle collider in the world.
When the LHC was first approved by the European Organization for Nuclear Research (CERN) in 1994, it was originally going to be built with an initial center-of-mass collision energy of \SI{10}{\tev}, with the plan to upgrade it to \SI{14}{\tev} at a later stage;
however, after negotiations with nonmember states, in 1996 the CERN council approved the construction at \SI{14}{\tev} energy in one stage \cite{doi:10.1146/annurev-nucl-102010-130438}.
First collisions were obtained in 2010 at center-of-mass energy of \SI{7}{\tev}, with the current world record of \SI{13}{\tev} being achieved in 2015 after the first Long Shutdown.

LHC is located at the CERN laboratory near Geneva, Switzerland, housed in the underground tunnel previously occupied by the LEP experiment.
Its structure, sketched in Figure \ref{fig:2:lhc_diagram}, consists of two counterrotating rings hosting beams for particle-particle collisions (mainly protons, but LHC is also used for ion collisions).

Four main experiments are stationed at the LHC ring intersection points:
ATLAS and CMS are high-luminosity experiments focused on general purpose proton-proton collisions; ALICE is optimized for lead-on-lead collisions with lower center-of-mass energy and luminosity compared to the former two; finally, LHCb is dedicated on the study of $b$ hadrons and will be the focus of the rest of this Chapter.
Beyond the above four, a number of small-scale, more specialized experiments also work with LHC, such as TOTEM, MilliQan and MoEDAL.

Broadly speaking, the LHC schedule alternates data taking periods (\textit{Runs}) with maintenance work periods (\textit{Long Shutdowns}, LS for short);
while the shutdowns are designed for consolidation and improvement of the collider itself, mainstay experiments usually take advantage of the hiatus to upgrade their detectors both in hardware and software.

@todo: discussione delle Run. Quello che segue va cambiato, quindi lo metto tradotto in italiano e tu ritraduci:

La Run 1 dell'LHC si è svolta dal 2010 alla fine del 2012. La Run 2 è iniziata a metà del 2015 e si è conclusa alla fine del 2018. Durante la Run 1 l'LHC ha fornito collisioni protone-protone con un'energia del centro di massa di 7 e 8 TeV, e in Run 2 l'energia è stata aumentata a 13 TeV.


\section{The LHCb experiment and detector}

\begin{figure}[t]
	\centering
	\includegraphics[width=\textwidth]{graphics/02-lhcb/lhcb_diagram.png}
	\caption[LHCb detector side view (Runs 1 and 2).]{Side view of the LHCb detector used for LHC Runs 1 and 2 \cite{Antunes-Nobrega:630827}.}
	\label{fig:2:lhcb_diagram}
\end{figure}

LHCb (the \textit{b} stands for \textit{beauty}\footnote{Before settling on the names \textit{top} and \textit{bottom} for the third generation of quarks, the names \textit{truth} and \textit{beauty} were among those proposed. While they never gained enough momentum in the scientific community, echoes of the failed nomenclature are still present in heavy quark vocabulary, for instance in the alternative name \textit{truth} for the \textit{topness} flavour number mentioned in Section \ref{sec:flavour-physics}, as well as in the official name for the LHCb experiment.}) is a single-arm detector designed to study heavy-flavour physics at the LHC, with the main objective of providing precision measurements of CP violation and rare decays of $b$ and $c$ hadrons \cite{Alves:1129809}.

Unlike the other three main experiments at LHC, LHCb has a forward-optimized geometry shown in Figure \ref{fig:2:lhcb_diagram}, with an angular acceptance of $10\div300$ \si{\mrad} in the bending plane and $10\div250$ \si{\mrad} in the non-bending plane\footnote{For the sake of brevity, I'll refer to it as the 300/250 \si{\mrad} acceptance.}.
Such a layout, more reminiscent of fixed target experiments than beam colliders, is motivated by the fact that $b\bar{b}$ pairs produced at high energies are usually collimated in the same forward/backward cone.
A more in-depth look at the tracking and particle identification systems will be taken in Sections \ref{sec:2:tracking} and \ref{sec:2:pid} respectively.
\label{info:LHCb_system}
The standard LHCb coordinate system, used as reference for the rest of this thesis, is a right-handed system centered on the beam interaction point, with the $z$ axis along the the beam pipe and $y$ axis directed vertically upwards.

@todo: Elenca i successi.

\subsection{Tracking}
\label{sec:2:tracking}
In order to measure the momenta of charged particles through their bending curve, LHCb employs a dipole magnet \cite{Amato:424338} consisting of two trapezoidal coils bent at $45^\circ$ on the two transverse sides, seen in Figure \ref{fig:2:lhcb_diagram} around $z\approx \SI{5}{\meter}$ (the magnet is placed so that the line connecting the centers of the pole faces crosses $z=\SI{5.3}{\meter}$).

\begin{figure}[t]
	\centering
	\includegraphics[width=.6\textwidth]{graphics/02-lhcb/b_field_map_z.png}
	\caption[LHCb magnetic field along the $z$ axis.]{LHCb magnetic field along the $z$ axis \cite{Amato:424338}.}
	\label{fig:2:b_field_map_z}
\end{figure}

This magnet provides an integrated field of $\int B dl \approx \pm \SI{4}{\tesla\meter}$ for \SI{10}{\meter} tracks\footnote{The $\pm$ sign is due to the fact that the magnet operates alternatively in up and down polarities, inverting the sign of the magnetic field.}.
Most of this field is contained in the $z\in[2.5,7.95]$ \si{\meter} region, with 
a small fraction ($\int B dl \approx \SI{0.12}{\tesla\meter}$) upstream of $z=\SI{2.5}{\meter}$. The field map along $z$, measured with a precision of $4 \times {10}^{-4}$, is shown in Figure \ref{fig:2:b_field_map_z} for $x=y=0$.
Dishomogeneities in the $xy$ plane for fixed $z$ are estimated at $\lesssim 6\%$ within the LHCb detector acceptance.

\subsubsection{VELO}
As the name suggests, the VErtex LOcator (VELO) system \cite{Barbosa-Marinho:504321} is designed to provide precision measurements of charged tracks near the beam interaction point, in order to correctly reconstruct detached secondary vertices typical of $b$- and $c$-hadron decays.

\begin{figure}[t]
	\centering
	\includegraphics[width=\textwidth]{graphics/02-lhcb/VELO_xy.png}
	\caption[Front view diagram of the VELO detector.]{Front view diagram of the VELO detector in fully closed (\textit{left}) and fully open (\textit{right}) configurations \cite{Barbosa-Marinho:504321}.}
	\label{fig:2:VELO_xy}
\end{figure}

The VELO detector comprises 42 silicon modules along the beam direction, each consisting of a pair of half discs measuring the radial and azimuthal track coordinates respectively. These modules cover the $1.6 < \eta < 4.9$ positive pseudorapidity range, as well as some negative pseudorapidity portion to improve primary vertex reconstruction, and are able to detect particles emerging from primary vertices with $|z| < \SI{10.6}{\centi\meter}$.
Due to high risk of radiation damage during beam injection from the Super Proton Synchrotron (SPS) into LHC, these modules can be retracted by \SI{3}{\centi\meter} in so-called \textit{fully open} configuration, whereas during collision phase the VELO operates in \textit{fully closed} configuration (see Figure \ref{fig:2:VELO_xy}).

\begin{figure}[t]
	\centering
	\includegraphics[width=\textwidth]{graphics/02-lhcb/VELO_xz.png}
	\caption[Cross section diagram of the VELO detector in the $xz$ plane.]{Cross section diagram of the fully closed VELO detector in the $xz$ plane at $y=0$ (top view). Radial sensors are depicted as \textit{solid} segments, azimuthal sensors as \textit{dashed} segments \cite{Barbosa-Marinho:504321}.}
	\label{fig:2:VELO_xz}
\end{figure}

Figure \ref{fig:2:VELO_xz} shows the $xz$ plane cross section of the VELO modules; the two halves of the detector are $z$-shifted by \SI{1.5}{\centi\meter} to ensure full azimuthal acceptance, resulting in the partial overlap seen in fully closed configuration.
Four radial-only \textit{pile-up sensors}, part of the Level-0 hardware trigger system (see Section \ref{sec:2:data_flow}), are placed upstream to help veto multiple-interaction events.



\subsubsection{Tracker Turicensis}
The Tracker Turicensis (TT) \cite{Gassner:728548}, formerly known as Trigger Tracker, is a $\SI{150}{\centi\meter} \times \SI{130}{\centi\meter}$ tracking station located just upstream of the dipole magnet.
Its placement serves the main purpose of tracking low-momentum particles ($|\vec{p}| \lesssim \SI{1.5}{\gev\per c}$) that would otherwise be bent out of the detector by the magnet without reaching the T stations.

The TT consists of four readout layers of silicon microstrip sensors arranged in a $x$-$u$-$v$-$x$ configuration (vertical in the first and last layers, rotated by a stereo angle of $\mp 5^\circ$ in the second and third layer respectively) for a total active area of $\approx \SI{8.4}{\meter\squared}$.
A \SI{200}{\micro\meter} strip pitch ensures a single-hit resolution $\lesssim \SI{50}{\micro\meter}$.

\begin{figure}[t]
	\centering
	\includegraphics[width=.6\textwidth]{graphics/02-lhcb/TT_layout.png}
	\caption[Front view of the third TT layer.]{Front view of the third TT layer (different readout sectors are labeled with different shadings) \cite{Alves:1129809}.}
	\label{fig:2:TT}
\end{figure}

The third TT layer is depicted in front view in Figure \ref{fig:2:TT}.
The basic unit of a layer is the \textit{half module}, covering half the LHCb height acceptance.
Each half module consists of a row of seven sensors bonded together to form either three or two \textit{readout sectors}.
Modules near the beam pipe are of the former category, with four sensors bonded in the L sector, two in the intermediate M sector and a single sensor for the K sector closest to the beam ($4$--$2$--$1$ modules);
other modules forgo the K sector and bond the spare sensor in the M sector ($4$--$3$ modules).
Front-end readout hybrids, one for each sector, are placed at the L-end of the half modules, outside of the detector acceptance, connected directly to the L sector and indirectly to the M and K sectors via Kapton flex cables.

\subsubsection{T stations}
\begin{figure}[t]
	\centering
	\begin{subfigure}{.45\textwidth}
		\includegraphics[width=\textwidth]{graphics/02-lhcb/t_station_top_view.png}
		\caption{}
		\label{fig:2:t_station_top}
	\end{subfigure}
	\begin{subfigure}{.45\textwidth}
		\includegraphics[width=\textwidth]{graphics/02-lhcb/t_station_front_view.png}
		\caption{}
		\label{fig:2:t_station_front}
	\end{subfigure}
	\caption[Top and front views of a T tracking station.]{Top (\textit{left}) and front (\textit{right}) views of a T tracking station \cite{Barbosa-Marinho:582793}. IT and OT are labeled with lighter and darker shades of grey respectively. Dimensions are given in \si{\centi\meter}; for the top view, lateral dimensions are not to scale.}
	\label{fig:2:t_station}
\end{figure}

The three T stations, labeled as T1--3, are the last line of defense for LHCb tracking purposes, covering the $z \approx 7.7 \div 9.4\,\si{\meter}$ region downstream of the dipole magnet \cite{Barbosa-Marinho:582793}.
Each T station is composed of an Inner Tracker for the region near the beam pipe and an Outer Tracker for the outer regions, as sketched in Figure \ref{fig:2:t_station}.

\begin{figure}[t]
	\centering
	\includegraphics[width=.6\textwidth]{graphics/02-lhcb/it_layout.png}
	\caption[Front view of an Inner Tracker $x$ layer.]{Front view of an $x$ detector layer in the T2 Inner Tracker \cite{Alves:1129809}.}
	\label{fig:2:IT}
\end{figure}

The Inner Tracker (IT) \cite{Barbosa-Marinho:582793} shares many similarities with the TT design, being developed in conjuction with it under the common Silicon Tracker (ST) project.
Sporting the same four layers of silicon microstrips in $x$-$u$-$v$-$x$ configuration, it covers a comparably smaller $\SI{120}{\centi\meter} \times \SI{40}{\centi\meter}$ cross-shaped surface (see Figure \ref{fig:2:IT}) for a total active area of $\approx \SI{4}{\meter\squared}$, less than half the TT.
As a consequence, individual modules only include one or two sensors  connected to the readout hybrids via a pitch adapter.

By contrast, the much larger Outer Tracker (OT) \cite{Barbosa-Marinho:519146} is a drift detector consisting in an array of Ar/CO$_2$ straw-tube modules.
Each module contains two layers of straw tubes with \SI{4.9}{\milli\meter} inner diameter, ensuring a \SI{50}{\nano\second} drift time and \SI{200}{\micro\meter} spatial resolution.
Within a single T station, said modules are arranged in four layers in $x$-$u$-$v$-$x$ configuration (see Figure \ref{fig:2:t_station_top}) with $\pm 5^\circ$ vertical tilt for $u$ and $v$ layers respectively.
The OT covers the entire 300/250 \si{\milli\rad} LHCb detector acceptance.

\subsubsection{Track classification and the problems with T tracks}
Overall, the LHCb tracking system has very high efficiency, besting $96\%$ in the momentum range $|\vec{p}| \in \left[5, 200\right]$ \si{\gev\per c} for tracks crossing all three detector stations (VELO, TT and T1--3). \cite{HistoryLHCb}
However, not all particles enjoy this luxury:
low momentum particles ($|\vec{p}| \lesssim \SI{1.5}{\gev\per c}$) are unable to reach the T stations due to the sharp magnet bending curve, while daughters of longer-lived particles with $c\tau \gtrsim \SI{30}{\centi\meter}$ will miss the VELO and possibly even the TT detector.

\begin{figure}[t]
	\centering
	\includegraphics[width=.8\textwidth]{graphics/02-lhcb/Track_Definitions.png}
	\caption[Side view diagram of LHCb tracking system and track categories.]{Side view diagram of the LHCb tracking systems for LHC Runs 1 and 2 with sketched examples of the main track classification categories.}
	\label{fig:2:track_classification}
\end{figure}

Thus, in spite of the great efficiency, it's useful to define track categories in the LHCb working environment depending on what hits were recorded in which detectors:
\begin{itemize}
	\item \textit{Long} tracks
	\item \textit{Upstream} tracks
	\item \textit{Downstream} tracks
	\item \textit{T} tracks
\end{itemize}
Sketches of tracks satisfying the above requirements are depicted in Figure \ref{fig:2:track_classification}.

@todo: \textbf{tutta} la storia delle T track. Le prime slide delle presentazioni, in pratica.

\subsection{Particle identification}
\label{sec:2:pid}
While tracking outgoing particles is obviously of paramount importance for physics analysis, knowledge of \textit{what} particles are being tracked is also crucial.
The ability to distinguish protons, pions and kaons is of particular interest at LHCb due to its research objectives in CP violation and $b$ physics, requiring precise flavour tagging and physical background rejection.
For the above reasons, a complex ecosystem of detectors dedicated to particle identification (PID) is in place.

\subsubsection{RICH}
Roughly $90\%$ of pions, protons and kaons from $B$ meson decays have momentum in the $[2,150]$ \si{\gev\per c} range \cite{HistoryLHCb}.
Since the momentum spectrum changes at different polar angles, LHCb employs two Ring Imaging CHerenkov (RICH) detectors \cite{Amato:494263} to cover the full momentum range for these particles.

\begin{figure}[t]
	\centering
	\begin{subfigure}{.45\textwidth}
		\includegraphics[height=.3\textheight]{graphics/02-lhcb/rich1_top.png}
		\caption{}
		\label{fig:2:rich_1_top}
	\end{subfigure}
	\begin{subfigure}{.45\textwidth}
		\includegraphics[height=.3\textheight]{graphics/02-lhcb/rich2_top.png}
		\caption{}
		\label{fig:2:rich_2_top}
	\end{subfigure}
	\caption[Top view of the two RICH detectors.]{Top view of the RICH 1 (\textit{left}) and RICH 2 (\textit{right}) detectors \cite{Alves:1129809}.}
	\label{fig:2:rich_top}
\end{figure}

The RICH 1 detector, sketched in Figure \ref{fig:2:rich_1_top}, is located upstream of the dipole magnet, wedged between the VELO and TT tracking detectors.
The detector exploits the different spectra of Cherenkov angles as a function of momentum for different kinds of particles.
During Run 1, RICH 1 used two radiator materials: an aerogel layer ($n=1.03$) and a C$_4$F$_{10}$ gas layer ($n=1.0014$).
This allowed RICH 1 to perform $\pi/K$ identification in the $1\div 60$ \si{\gev\per c} range.
Due to occupancy problems, the silica aerogel radiator providing identification in the low momentum range $|\vec{p}| \lesssim \SI{10}{\gev\per c}$ was removed for Run 2;
since the kaon Cherenkov threshold in C$_4$F$_{10}$ is $\approx \SI{9.7}{\gev\per c}$, they can still be identified by operating RICH in so-called \textit{kaon veto mode}, i.e. 
by the lack of Cherenkov light \cite{HistoryLHCb} \cite{RichPerformance}.
RICH 1 covers from $\SI{25}{\mrad}$ (lower limit imposed by the beryllium beam pipe section) up to the full 300/250 \si{\mrad} LHCb acceptance.

Acting as complement to its partner, RICH 2 (Figure \ref{fig:2:rich_2_top}) operates downstream of the T tracking stations and is optimized for a high momentum range, providing PID from $\approx \SI{15}{\gev\per c}$ up to and beyond $\SI{100}{\gev\per c}$.
Its lower limit of acceptance is $\approx \SI{15}{\mrad}$, dictated by the required clearance of \SI{45}{\milli\meter} around the beam pipe.

\subsubsection{Calorimeter}
\begin{figure}[t]
	\centering
	\begin{subfigure}{.45\textwidth}
		\includegraphics[width=\textwidth]{graphics/02-lhcb/ecal_segm.png}
		\caption{}
		\label{fig:2:ecal_segm}
	\end{subfigure}
	\begin{subfigure}{.45\textwidth}
		\includegraphics[width=\textwidth]{graphics/02-lhcb/hcal_segm.png}
		\caption{}
		\label{fig:2:hcal_segm}
	\end{subfigure}
	\caption[Front view of the lateral segmentation of SPD/PS, ECAL and HCAL calorimeters.]{Front view of the lateral segmentation of SPD/PS and ECAL (\textit{left}) and HCAL (\textit{right}) calorimeters \cite{Alves:1129809}. Only a quarter of the detector is depicted. Dimensions are given for the ECAL in the left figure.}
	\label{fig:2:ecal_hcal_segm}
\end{figure}

The LHCb calorimeter system \cite{Amato:494264} serves the dual purpose of identifying hadrons, electrons and photons and measuring their energies.
Its design follows the standard high energy physics approach of an electromagnetic calorimeter (ECAL) for the detection of electrons and photons, followed by a hadronic calorimeter (HCAL) for the detection of charged and neutral hadrons.

Placed at \SI{12.5}{\meter} from the beam interaction point, the ECAL employs a shashlik layout\footnote{The nomenclature references the \textit{shashlik}, or \textit{šašlyk}, a traditional meat dish consisting of skewers threaded with alternating pieces of meat, fat and vegetables. The dish is popular throughout the Caucasus and Central Asia regions, including the former Soviet Union, where the shashlik calorimeter technology was first developed.}, alternating layers of absorber (\SI{2}{\milli\meter} thick lead) and sampler (\SI{4}{\milli\meter} thick polystyrene scintillator tiles) perpendicular to the beam axis.
Due to the steep dependence of hit density from the distance from the beam pipe, the calorimeter adopts a variable cell size and is segmented in thee distinct sections outlined in Figure \ref{fig:2:ecal_segm}.
The ECAL is roughly $25X_0$ long, with $X_0$ being the radiation length; this allows for the full containment of electromagnetic showers from high energy photons, which is of paramount importance for energy resolution.

Electron detection is particularly tricky due to the significant pion background, both of the charged and neutral variety.
To combat this, two ancillary detectors are located upstream of the ECAL proper: the scintillator pad detector (SPD) selects charged particles to veto $\pi^0$, while the preshower detector (PS) rejects $\pi^\pm$.
Collectively, the SPD/PS detectors consist in two scintillator pads enclosing a \SI{15}{\milli\meter} thick lead plate with a $\SI{7.6}{\meter} \times \SI{6.7}{\meter}$ sensitive area.
Transverse segmentation is designed to projectively match the ECAL segmentation down to the individual cell size.

The HCAL is a sampling calorimeter as well, employing iron as absorber and scintillating tiles as active material.
In contrast to the ECAL and SPD/PS detectors, however, the scintillating tiles run parallel to the beam axis, interspersed with \SI{1}{\centi\meter} thick layers of iron;
meanwhile, the longitudinal structure alternates scintillating tiles with iron spacers, both of length $\lambda_I \approx \SI{20}{\centi\meter}$, $\lambda_I$ being the hadron interactiorn length in steel.
The transvese segmentation of the detector, sketched in Figure \ref{fig:2:hcal_segm}, envisages one less section and a comparably larger cell size than ECAL, owing to the differing sizes of electromagnetic and hadronic showers.
Since hadron energy resolution does not require full containment of the shower, the HCAL only extends for $\approx 5.6$ interaction lengths.

In all four subdetectors, scintillating light is conveyed through wavelength-shifting fibres to photomultiplier tubes for conversion and magnification; due to the lower light yield of HCAL modules, their phototubes operate at a higher gain.

\subsubsection{Muon system}

The final components of the PID system are the five muon stations M1--5 \cite{Barbosa-Marinho:504326}, providing trigger and limited tracking for muons in LHCb.
Stations M2--5 are placed downstream of the calorimeter system, separated between each other by \SI{80}{\centi\meter} of iron;
these absorber layers select muons on the basis of penetration, with a \SI{6}{\gev\per c} momentum threshold required to cross the fifth station.
The lone M1 station precedes the calorimeter system with the goal of improving transverse momentum measurement.
The muon system provides acceptance in the $20 \div 306$ \si{\mrad} region in the bending plane and $16 \div 258$ \si{\mrad} region in the non-bending plane, in line with the global LHCb acceptance.

\begin{figure}[t]
	\centering
	\includegraphics[width=.5\textwidth]{graphics/02-lhcb/muon_side_view.png}
	\caption[Side view diagram of the muon system.]{Side view diagram of the muon system \cite{Alves:1129809}.}
	\label{fig:2:muon_side_view}
\end{figure}


A side view diagram of the muon system is depicted in Figure \ref{fig:2:muon_side_view}: each station is divided in four R1--4 regions with increasing distance from the beam pipe.
While transverse spatial resolution progressively worsens in outer regions, the growing influence of large angle multiple scattering means it would be limited anyway.

The most sensitive area is the R1 region of the M1 station, since the large particle flux imposes strict limits on radiation hardness to prevent ageing effects during the LHC projected lifetime.
For this reason the M1-R1 region alone employs gas electron multiplier foils, while the remainder of the muon system consists of multi-wire proportional chambers with a Ar/CO$_2$/CF$_4$ gas mixture.

Overall, the five stations combined cover a total area of \SI{435}{\meter\squared}. Stations M1--3, by virtue of their high spatial resolution along the $x$ coordinate, are used to determine the direction of the candidate muon track and compute the transverse momentum with $\approx 20\%$ resolution;
stations M4--5 have lower performance on this front and their contribution mainly consists in the identification of highly penetrating particles.

\section{The LHCb data flow}
\label{sec:2:data_flow}

\begin{figure}[t]
	\centering
	\includegraphics[width=\textwidth]{graphics/02-lhcb/lhcb_run_1_data_flow.png}
	\caption{Diagram of the LHCb Run 1 data flow.}
	\label{fig:2:lhcb_data_flow}
\end{figure}


@todo: introduzione

\subsection{Trigger}
The trigger system \cite{Antunes-Nobrega:630828} provides the first triage of all data recorded by the LHCb detector.
LHC collides proton-proton bunches at a nominal \SI{40}{\mega\hertz} rate, with $\approx 1\%$ resulting in $b\bar{b}$ events of interest for LHCb.
Furthermore, only $\approx 15\%$ of these events will produce a reconstructible $b$ hadron (i.e. with all decay products within the detector acceptance) \cite{HistoryLHCb}, and studies on topics such as CP violation are likely to require decays with small ($\lesssim {10}^{-3}$) branching ratios.
Peak writing speeds for data storage are in the order of a few \si{\kilo\hertz}, making it impossible to save all information even forgoing the high costs this would entail in terms of storage space.
The LHCb trigger system therefore has to skim out the vast majority of uninteresting events with high efficiency, and it needs to be fast about it.

The trigger system employed during LHC Runs 1 and 2 can be broken down into three distinct phases, or \textit{levels}.
First comes Level-0 (L0), which is implemented directly on hardware via custom-made electronics.
Working synchronously with the \SI{40}{\mega\hertz} bunch-crossing rate, the L0 trigger is only able to read parts of the LHCb detector independently: 

\subsection{From reconstruction to analysis}

\subsection{Monte Carlo simulations}



\section{LHCb detector upgrade for Run 3}

\begin{figure}[t]
	\centering
	\includegraphics[width=\textwidth]{graphics/02-lhcb/lhcb_diagram_run3.png}
	\caption[LHCb detector side view (Run 3).]{Side view of the upgraded LHCb detector for future usage in LHC Run 3 \cite{Piucci_2017}.}
	\label{fig:2:lhcb_diagram_run3}
\end{figure}

\section{Data used for this thesis}
Non so se vada qui ma da qualche parte deve andare.