\chapter{The Standard Model of Particle Physics}

Ever since Democritus' philosophy of atomism, one of the driving desires behind mankind's advancements in the fields of natural science has been to reduce reality to its basic components.

[...], this chapter explores the theoretical framework for the rest of thesis, introducing key concepts such as \textit{spin}, \textit{helicity} and the importance of \textit{angular distributions} of decay products.

\section{Elementary particles}
\begin{figure}[t!]
	\centering
	\includegraphics[scale=0.15]{graphics/01-standard_model/Standard_Model_of_Elementary_Particles.pdf}
	\caption[Currently known Standard Model elementary particles.]{The seventeen currently known elementary particles of the Standard Model. Antiparticles are not depicted.}
	\label{fig:particle_zoo}
\end{figure}

Intuitively, a particle is said to be \textit{elementary} when no substructure can be probed. 
A century of efforts in the fields of nuclear, quantum, and high energy physics has whittled down the spectrum of matter to just seventeen unique fundamental particles, colloquially known as the \textit{particle zoo} and depicted in Figure \ref{fig:particle_zoo}.

Each particle is joined by an \textit{antimatter particle} (\textit{antiparticle} for short), a companion of opposite charge identified by the prefix \textit{anti-}, e.g. antimuon for the muon; the only exception to this naming convention is the electron, whose antiparticle, for historical reasons, is known as positron.
While often omitted for the sake of brevity, antiparticles are elementary particles in every respect, distinct from their partners (bar the neutral gauge bosons, which are their own antiparticles) and related to them through the transformation of \textit{charge conjugation}.

\subsection{Leptons}
Fermioni. Neutrini privi di massa ma non lo sono davvero.

\subsection{Quarks}
Adroni, mesoni e barioni. QCD. Storia della scoperta. Tre colori. Mescolamento dei sapori.

Unlike leptons, quarks are impossible to observe directly: according to the phenomenon of \textit{color confinement}, the energy of the interaction field between two color charges being pulled apart increases with their distance until it becomes high enough to create a quark-antiquark pair.
This process of \textit{fragmentation} develops many times over in such a way that the final observable state is entirely composed of colorless particles.
For this reason, high energy physics experiments such as LHCb do not detect free quarks, instead observing cone-shaped streams of hadrons known as \textit{hadronic jets}.

\subsection{Gauge bosons and fundamental interactions}
The fundamental forces driving the interactions between elementary particles are introduced in the Standard Model via the so-called \textit{gauge principle}.

[Gruppo di simmetria, principio di gauge, QCD e teoria elettrodebole.]

There is no gauge boson associated to the fourth known fundamental force, gravity.
Since every attempt to reconcile the general theory of relativity with quantum mechanics has failed so far, gravity is presently excluded from the Standard Model; this doesn't affect SM predictions at the subatomic level on account of the remarkably low intensity of said force, over 30 orders of magnitude lower than the weak interaction.

\subsection{The Higgs boson}
Rottura spontanea della simmetria.

\section{Spin}
Spin ed EM dipole.

\section{Discrete symmetries}
CPT, violazione di CP.

\section{Helicity formalism}
Il Richman.