\chapter*{Acknowledgements}
\addcontentsline{toc}{chapter}{Acknowledgements}

The work presented in this thesis would not have been possible without steady input and support from the Milan and Valencia LHCb research groups.
While it would be impossible for me to recount every contribution made, I struggle to think of a single person who did not volunteer to help me whenever I needed it. 

Although I never had the pleasure of working alongside him directly, Salvatore Aiola laid most of the groundwork for the \demonstratorshort signal selection process and his influence is ubiquitous.
The foundation in such a sprawling analysis is often the most critical step, and I couldn't have asked for a better bedrock than the one he left me.
%Most of my contributions were made tracing the footsteps of Salvatore's groundwork, especially concerning the \bz invariant mass veto.
His and Luca Pessina's comparative work on different multivariate classifiers was also instrumental in the choice of the histogrammed boosted decision tree I ended up using.

The treatment of ghost vertex \lambdadecay events owes a lot to Joan Ruiz Vidal's identification and systematization of the problem.
Without his intuition and $\psi$--$h$ parameterization of the ghost vertex locus of points, I would not have gained nearly as much insight into the nature of the issue and its impact on proton angular resolution.

Contributions by Andrea Merli and Giorgia Tonani cannot be overstated.
Along with my supervisor and co-supervisor, they were the ones who molded the inexperienced physics student I was at the beginning of this year-long endeavour into the slightly more competent soon-to-be graduate I am now;
without them, I doubt I would have achieved half of what is included in this thesis.
Giorgia's excellent study of the \lz polarization from the $\Xi_c^0 \rightarrow \Lambda^0 K^- \pi^+$ decay was a beacon of light during the drafting of Chapter \ref{cap:angular_distribution} and spared be several sleepless nights.

Finally, special thanks are due to Fernando Martinez Vidal, with whom I worked closely in the development of the Armenteros-Podolanski veto and who went above and beyond in his guidance throughout.